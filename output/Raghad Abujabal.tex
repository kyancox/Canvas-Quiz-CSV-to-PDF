\documentclass[11pt]{article}
\usepackage[utf8]{inputenc}
\usepackage[margin=1in]{geometry}
\usepackage{amsmath}
\usepackage{amssymb}
\usepackage{enumitem}
\usepackage{microtype}
\usepackage{ragged2e}

% Better handling of line breaks
\sloppy
\emergencystretch=1em

\title{Quiz Submission}
\author{Raghad Abujabal}
\date{}

\begin{document}

\maketitle

\noindent\textbf{Student ID:} 385605

\vspace{1em}

\section*{Question 1}

\begin{small}
1. SuperSort(A) { \\ sorts array A recursively \\
 \\
2. // base cases \\
 \\
3. if (length(A) == 1) return; \\
 \\
4. if (length(A) == 2) { \\
 \\
5. swap the two elements if they’re out of order; \\
 \\
6. return; \\
 \\
7. } else { // recursive calls \\
 \\
8. SuperSort the first two thirds of A; \\
 \\
9. SuperSort the second two thirds of A; \\
 \\
10. SuperSort the first two thirds of A again; \\
 \\
11. } \\
 \\
12. }
\end{small}

\vspace{0.5em}
\noindent\textbf{Answer:}

\begin{RaggedRight}
so if length a = 1, we will return, if it's 2, we will swap elements(base), and then we will do 3 recursive calls on the first 2/3 of a, then 2nd, then 3rd. \\ therefore, Our recursive solution would be T(n)=3T(2n/3)+O(n).
\end{RaggedRight}

\vspace{1em}

\section*{Question 2}

Part B: Give a recurrence relation upper bound, tight up to constant factors, for the performance of your algorithm given in Part A. Don't forget the base case(s).

\vspace{0.5em}
\noindent\textbf{Answer:}

\begin{RaggedRight}
T(n)=T(k/2)+O(1)
\end{RaggedRight}

\vspace{1em}

\section*{Question 3}

Part C: Solve your recurrence relation given in Part B by providing an asymptotic solution tight up to constant factors, but do NOT use the master theorem or the master-master theorem/nuclear bomb. Show your work.

\vspace{0.5em}
\noindent\textbf{Answer:}

\begin{RaggedRight}
the solution would just be adding the constant factors,
\end{RaggedRight}

\vspace{1em}

\section*{Question 4}

Suppose you are given positive integers where for nonnegative integer . You would like to determine using only the elementary operations of addition, subtraction, multiplication, and division.

A brute force algorithm is described below.



The brute force algorithm requires time to find the correct value of . To help with this task, your friend designed a function called which takes input of the form and computes in time. In Part A, you will use to design a divide and conquer algorithm to find the correct value of with smaller asymptotic runtime than the above brute force algorithm. Hint: , so such that if is even or if is odd.

Part A: Design a divide and conquer algorithm that has an asymptotically faster performance than the given brute force algorithm.

\vspace{0.5em}
\noindent\textbf{Answer:}

\begin{RaggedRight}
base : if a =1, then k=
0. our design is, If let's say that we have a variable n and n = sqrta so the means that n = $b^k/$
2. we want to decide if k is even or odd, if it's even then n would be $(b^k/2)$ so that makes $n^2$ then k= 2(k/2) then if it's odd then a = b multiplied by b*$(b^k/2)^{2}$ and that would make k= 2(k/2)+
1. then the runtime would be O(log k)
\end{RaggedRight}

\vspace{1em}


\end{document}

